\chapter{Bevezetés}

\section{Előző féléves munkám}

A Budapesti Műszaki és Gazdaságtudományi Egyetem Villamosmérnöki és Informatikai karán a villamosmérnök alapszak 5. félévében Témalaboratórium tantárgyat hirdetnek.
Ez a tantárgy egy olyan projekttárgy, amely a hallgatók számára lehetővé teszi, hogy valamilyen területen elmélyedjenek, és közösen dolgozzanak egy témavezetővel, ezzel tapasztalatot nyújtva számukra, a későbbi Önálló laboratórium és Szakdolgozat készítés tárgyakhoz.

Én az 5. félévemben a Témalaboratórium keretei között Dr. Horváth Bálint Péter témavezetésével folytattam egyéni munkát, amely során egy OFDM (Orthogonal Frequency Division Multiplexing - Ortogonális Frekvenciaosztásos Nyalábolás) rendszert modelleztem. A munka során Python nyelvben készítettem el egy digitális jelátvitel rendszer modelljét, amelyben az adót, a vevőt, és köztük az átviteli csatorna volt a 3 legfőbb elem. Munkám során azt vizsgáltam, hogy miképp tud egy ilyen OFDM rendszer védekezni a többutas terjedés problémái ellen, milyen lehetőségek vannak a vett jel kompenzálására, vagy az átvitel minőségének mérésére.

Az eredmények részletes tárgyalása nem része ennek a dolgozatnak, de az eredményeket is és a tanultakat is felhasználtam a féléves munkámban.

\section{OFDM és OFDMA}

Az OFDM és az OFDMA (Orthogonal Frequency Division Multiple Access) rendszerek között egy apró különbség van. Míg egy OFDM rendszerben egy adó és egy vevő közötti kommunikáció valósul meg több vivőfrekvencián, addig az OFDMA rendszerben a kommunikáció egy vagy több adó és egy vagy több vevő között valósul meg. Ez számomra nem volt evidens, a tanulmányaim során eddig a kettőt összemosták.

Az OFDM előnyeit könnyedén kihasználhatjuk  

\section{OFDMA kihívásai}

xdxd